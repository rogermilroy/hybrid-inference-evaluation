\documentclass[]{article}
\usepackage{graphicx}
\usepackage{hyperref}

\begin{document}

\title{Applications of Navigation in GPS Denied Environments}
\date{2019\\ November}
\author{Roger Milroy\\ Department of Computer Science\\ Royal Holloway, University of London}

\maketitle

\section{Introduction}

This report is aimed at exploring some of the real world implications of the project that I am carrying out. It will also form the core of the Professional Issues section of my final report.
I will outline the key advantages that being able to navigate drones effectively without GPS has across all applications. 
I will briefly outline some potential positive applications as well as some of the potential negative applications.
I will finally attempt to bring the discussion back to the current state of reality and come to some conclusion about the benifits, risks and where to go from here.


\section{Key Advantages}

\subsection{Enhanced reliability}

Most current navigation solutions rely on Satellite Positioning Systems in order to fix their location.
Almost all people now regularly use these systems on their mobile phones or in their cars. 
Most people would know them by the acronym GPS but there are now 4 separate systems some of which can be used by the same or similar receivers and others that are incompatible.
These systems are GPS (US), GLONASS (Russia), Galileo (EU) and BDS (China).

Most systems can use 2 or 3 of these which does help add to the reliability of these systems, if one has problems or there are insufficient satellites in view of the receiver, you can rely on another system.
However there are still potential issues of reliability. Most thankfully are rare occurrences. 
These broadly fall into two categories. 
Extreme space weather events and Bad Actor attacks. 
Coronal Mass Ejections, also known as X-class solar flares are eruptions from the surface of the Sun that can seriously disrupt not only transmissions but also satellites electronics. 
These are rare and usually forecast so damage is unlikely but coverage is likely to be interrupted for a sufficiently large one.
There are a couple of potential attacks on satellite systems, only the most advanced are aimed at the satellite constellations directly and these are only likely in the event of war.
Lesser attacks would be more along the lines of spoofing or blocking GPS signals. With the effect of either denying coverage or misleading the target as to their location.

Obviously all of the above, though infrequent or unlikely could have a serious impact if they were to happen at the wrong time.
Having an effective back up solution may make a serious impact in certain scenarios.

\subsection{Removal of Environmental Restrictions}

The main issue however is the large class of environments where GPS reception is anything less than perfect. 
These range from areas with restricted or challenging signal, such as in Forests. Depending on the density of the tree cover it may be possible to get some signal, to have intermittant signal or even to have none at all in the worst case.
There is a large number of environments where GPS signal is essentially impossible. Many of these are where most of us spend the majority of our time. Indoors.
For any autonomous device that is designed to be operated indoors, it needs some GPS independant navigation system.
Along the same lines in terms of signal is underwater. GPS signals cannot penetrate water for any appreciable depth so it is essentially a no-GPS zone. 
Any level of autonomous behaviour needs a reasonable independant navigation solution.

\section{Application Areas}

Now that I have established why reliance on GPS is detrimental and indicated some areas where it is either impossible or challenging, I will now explore in more depth the potential applications of non GPS reliant navigation.

\subsection{Positive}

There are many excellent applications that can be thought of where navigation in these environments is important.
One of the most obviously positive applications is in the area of search and rescue. This often takes place in unpredictable environments where there are no guarantees of GPS signal. Think of search in forested areas, or inside damaged buildings.
Both of these could clearly be helped by the deployment of autonomous vehicles either to increase search area or to navigate potentially unstable structures.
A key advantage would be that it would enable greater flexibility of deployment. Moving from GPS enabled to denied with minimal issues would open up mission profiles and increase usefulness.
A similar situation could be to evaluate damaged structures from the inside, before sending in human inspectors it would be preferential to reduce the risk by a preliminary investigation.
As mentioned before almost all sub sea drone activity needs a GPS independant navigation system, this could be for exploration, to map the sea floor more accurately for example or for more specific investigatory work, such as exploring shipwrecks.



\subsection{Negative}

There are a few key areas that would probably be classed at best as questionable and at worst as strongly negative. 
These are usually tied up with military concerns or surveillance concerns. There are arguments to be had about the extent to which these are fully or partly negative but 
most people would at least agree that these are areas that require care and could be negative.

With that said, these are the areas that could benefit from this technology that I would consider to be negative.
Autonomous or semi autonomous military drones that need to operate indoors or in say cave systems, forests etc. 
A reliable navigation solution would open up these areas for operation. This can be a good thing, think of bomb disposal robots that could be made semi autonomous.
On the flip side, there is the worrying situation of militarised robots operating in peoples homes. Though it may seem distant, remember that urban combat is one of the most dangerous forms of combat. 
And that governments are very sensitive to military deaths nowadays. It follows, that it would be extremely interesting for governments to be able to take their soldiers out of harms way in this manner.

Underwater drones as mentioned before will be enabled and have many positive applications. However it may also enable military submarine drones that could at a relatively low cost be used to deny access to shipping lanes in the worst cases.

And all of these are without mentioning the threat of terrorism. As we have seen in recent years, small organisations can use drones for military effect, think of the supposed attack in Venezuela where drones were blamed.

Or think of the denial of use of Gatwick by what didn't seem to be autonomous drones. The same problem would be harder to deal with if they were autonomous. Jammers would be ineffective so direct kinetic approaches would be necessary.
If the drones relied on GPS it is possible to interfere with them as I mentioned before by spoofing or denial. With reliable and accurate on board navigation systems that is ineffective and so kinetic solutions would be the only answer.

\section{Conclusion}

To close I want to consider how my project directly impacts the situation if at all.

Firstly I must admit that there are already similar solutions in most mid level commercial drones. DJI in particular has a module specifically for non satellite dependant positioning.
Secondly my project is aimed at improving the accuracy of existing techniques and technologies but current accuracy is already high enough to effectively navigate in all but the most extreme of environments.

And finally and probably the most important is that the computational cost of my project is much higher than existing techniques. It is therefore unlikely to enable any smaller applications and in fact may be limited to very large commercial drones.


%%%% ADD YOUR BIBLIOGRAPHY HERE
\bibliographystyle{acm}
\bibliography{../resources/final_project}



\end{document}

\end{article}
